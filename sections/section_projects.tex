\vspace*{-5mm}
\sectionTitle{Projects}{\faLaptop}
\begin{projects}
\whenrole{ai}{
\project
{Technical Documentation RAG System}{Jan 2025 - Present}
{\website{Github}{https://github.com/apassuello/technical-doc-rag/} | \website{Demo}{https://huggingface.co/spaces/ArthyP/technical-rag-assistant}}
{Architected production-grade RAG system achieving 99.5\% chunk quality through novel hybrid parsing combining TOC-guided navigation with PDFPlumber extraction.
Implemented hybrid retrieval system fusing dense (FAISS) and sparse (BM25) search with configurable RRF fusion, reducing query latency to <100ms.
Developed multi-model answer generation supporting Ollama (Llama 3.2), HuggingFace API, and InferenceProviders with adaptive prompt engineering.
Optimized embedding pipeline for Apple Silicon MPS acceleration, achieving 129.6 texts/second throughput—6x faster than baseline CPU implementation.
Built comprehensive evaluation framework with 67 unit tests, integration testing, and manual quality assessment revealing critical improvements over naive approaches.}
{PyTorch, FAISS, BM25, Llama 3.2, Streamlit, Docker, RAG, NLP, Hybrid Search, PDF Processing, Sentence Transformers, HuggingFace Deployment, MPS Acceleration, Prompt Engineering, RRF Fusion, RISC-V, Technical Documentation, Production Deployment, pytest}
}
\whenrole{consulting}{
\project
{Technical Documentation RAG System}{Jan 2025 - Present}
{\website{Github}{https://github.com/apassuello/technical-doc-rag/} | \website{Demo}{https://huggingface.co/spaces/ArthyP/technical-rag-assistant}}
{Architected production-grade RAG system achieving 99.5\% chunk quality through novel hybrid parsing combining TOC-guided navigation with PDFPlumber extraction.
Implemented hybrid retrieval system fusing dense (FAISS) and sparse (BM25) search with configurable RRF fusion, reducing query latency to <100ms.
Developed multi-model answer generation supporting Ollama (Llama 3.2), HuggingFace API, and InferenceProviders with adaptive prompt engineering.
Optimized embedding pipeline for Apple Silicon MPS acceleration, achieving 129.6 texts/second throughput—6x faster than baseline CPU implementation.
Built comprehensive evaluation framework with 67 unit tests, integration testing, and manual quality assessment revealing critical improvements over naive approaches.}
{PyTorch, FAISS, BM25, Llama 3.2, Streamlit, Docker, RAG, NLP, Hybrid Search, PDF Processing, Sentence Transformers, HuggingFace Deployment, MPS Acceleration, Prompt Engineering, RRF Fusion, RISC-V, Technical Documentation, Production Deployment, pytest}
}
\project
{MultiModal Insight Engine\whenrole{firmware}{ (WIP)}}{Feb 2025 - Present}
{\website{Github}{https://github.com/apassuello/multimodal\_insight\_engine/}}
{Designed and implemented a full-stack training and evaluation platform for Transformer-based multimodal models.
Developed modules for mixed precision, quantization, and pruning; integrated benchmarking tools to compare performance across optimizations.
Built tokenizers and dataloaders for multilingual datasets\whenrole{general}{ (WMT, Europarl, IWSLT)}; supported joint BPE and custom preprocessing.
Integrated pretrained models \whenrole{general}{(CLIP, ViT) }with a unified interface for image-text tasks; implemented custom loss functions and schedulers.
Conducted safety evaluations via red teaming frameworks, adversarial generation, and prompt injection testing.}
{PyTorch, Transformers, NLP, Vision Models, Model Optimization, Tokenization, CLIP, ViT, Mixed Precision, Quantization, Pruning, Red Teaming, BLEU/F1/Accuracy, t-SNE, Config Management, Visualization, Logging, Safety Evaluation}
\whennotrole{executive}{
\whenrole{general}{
\newpage
}
\project
{ASIC-Enabled Medical Device Development}{2021 - 2022}
{\website{ADEPT Neuro SA}{https://adeptneuro.com}}
{Development of hardware-software interfaces for novel deep-brain electrodes with integrated ASICs\whenrole{firmware}{, demonstrating embedded hardware design expertise and complex system integration}. Implementation of machine learning-based seizure detection system, achieving excellent academic recognition (5.0/6.0). Project focused on \whenrole{firmware}{deeply integrated, high-performance }medical device \whenrole{firmware}{platforms within startup environment}\whenrole{general}{innovation within a startup environment at EPFL Innovation Park}.}
{Hardware-Software Integration, ASIC Integration, Machine Learning, Medical Devices, C++, Python}
}
\whenrole{general}{
\project
{Data-Driven Analysis of DeepWeb Marketplace Dynamics}{2019}
{\website{EPFL ADA Course}{https://github.com/ADA-2019/Project}}
{Large-scale data analysis project processing over 1TB of historical marketplace data (2013-2015) to study behavioral economics and regulatory impact. Implemented comprehensive data processing pipeline and statistical analysis to evaluate short-term versus long-term marketplace dynamics. Project received distinguished academic recognition (5.5/6.0).}
{Python, Data Analysis, Big Data Processing, Statistical Analysis, Web Scraping, Pandas}
}
\whenrole{firmware}{
\project
{Data-Driven Analysis of DeepWeb Marketplace Dynamics}{2019}
{\website{EPFL ADA Course}{https://github.com/ADA-2019/Project}}
{Large-scale data analysis project processing over 1TB of historical marketplace data (2013-2015) to study behavioral economics and regulatory impact. Implemented comprehensive data processing pipeline and statistical analysis to evaluate short-term versus long-term marketplace dynamics. Project received distinguished academic recognition (5.5/6.0).}
{Python, Data Analysis, Big Data Processing, Statistical Analysis, Web Scraping, Pandas}
\newpage
\project
{High-Performance Genomic Processing}{2018}
{\github{HEIG-VD-Genomics/FM-Index-FPGA}}
{Implementation of FM-Index algorithm for genomic sequence matching on Artix-7 FPGA platform with HMC memory integration. Developed complete embedded hardware architecture and comprehensive testing infrastructure, achieving performance metrics surpassing traditional software solutions through optimized hardware-software integration. Project earned exceptional recognition (5.5/6.0).}
{FPGA, Verilog, HMC Memory, Genomics, Hardware Architecture, System Testing}
}
\whenrole{general}{
\project
{PriorityQueue Formal Verification}{2019 - 2020}
{\website{}{https://github.com/fquellec/PriorityQueue-FormalVerification}}
{An attempt at proving the correctness of a Stainless implementation of theOptimal Purely Functional Priority Queues introduced by Chris Okasaki and Gerth Stølting Brodal (6.0/6.0)}
{Scala, Stainless, Formal Verification}
}
\end{projects}